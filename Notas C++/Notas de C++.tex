\documentclass[]{article}
\usepackage[utf8]{inputenc}
\usepackage[spanish]{babel}
\usepackage{amsmath}
\usepackage{geometry}
\usepackage{multicol}
\usepackage{amssymb}
\usepackage{hyperref}

\geometry{a4paper,left=3cm,right=3cm,top=3cm,bottom=3cm}

%opening
\title{Notas de C++}
\author{Logan Martinez}

\begin{document}
	
	\maketitle
	
	\begin{abstract}
		Estas son las notas sobre mi aprendizaje de C++
	\end{abstract}

	\section{Introducción}
	C++ Es un lenguaje de programación basado en C con una buena cercanía a la maquina, con un desarrollo mucho mas veloz; tiene la ventaja de ser un lenguaje con elementos de programación orientada a objetos que compila directamente a binario, siendo que originalmente compilaba en C puesto a que se consideraba una extensión de este lenguaje.
	
	\section{Una sobre vista de C++}
	
	C++ es un lenguaje de programacion orientado a objetos. antes de aprender algo en especifico sobre C++, es importante entender la teoria basica detras de la programacion orientada a objetos.
	
		\subsection{Que es la programacion orientada a objetos?}
		
			\subsubsection{Objetos}
		
			La mas importante caracteristica de un lenguge orientado a objetos es el objeto. Ponlo simple, un \textit{objeto} es una entidad logica que contiene datos y codigo que manipula esos datos. En el objeto, parte del codigo y/o los datos pueden ser privados para el objeto e inaccesible directamente desde cualquier cosa fuera del objeto. En esta forma un objeto brinda un significativo nivel de proteccion contra alguna otra parte parte del programa de accidentalmente modificar o incorrectamente utilizar partes privadas del objeto. La union entre el codigo y los datos es denominada \textit{Encapsulacion}
		
			Para todas las intenciones y propositos, un objeto es una variable de un tipo definido por el usuario.
		
			\subsubsection{Polimorfismo}
			
			Lenguajes orientados a objetos soportan \textit{polimorfismo}, que esencialmente significa que un nombre puede ser usado para varios relacionados, pero ligeramente diferentes propositos. El proposito de polimorfismo es el permitir a un nombre ser usado para especificar una clase general de acciones.
			
			\subsubsection{Herencia}
			
			La \textit{Herencia} es el proceso por el que un objeto puede adquirir las propiedades de otro objeto. Esto es importante porque soporta el concepto de clasificación 
			
			
			
			
\end{document}